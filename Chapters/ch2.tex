\chapter{常用}

\section{图片}

\subsection{单图居中}
如图~\ref{fig::graphexample}所示
\begin{figure}[htbp]
	\centering
	\includegraphics[width = 0.9\textwidth]{./Graphs/example/example.jpeg} 
	\caption{单栏居中图片}
	\label{fig::graphexample} 
\end{figure}
\subsection{双列跨页子图}

\begin{figure}[!ht] \centering
	\begin{subfigure}[b]{0.49\textwidth}\centering
		\includegraphics[width=\textwidth]{./Graphs/example/example.jpeg}
		\caption{sub1}\label{subf::sub1}
	\end{subfigure}\hfill % 插入间隔
	\begin{subfigure}[b]{0.49\textwidth}\centering
		\includegraphics[width=\textwidth]{./Graphs/example/example.jpeg}
		\caption{sub2}\label{subf::sub2}
	\end{subfigure}
\end{figure}
\begin{figure}[!ht]\ContinuedFloat \centering
	\begin{subfigure}[b]{0.49\textwidth}\centering
		\includegraphics[width=\textwidth]{./Graphs/example/example.jpeg}
		\caption{sub3}\label{subf::sub3}
	\end{subfigure}\hfill % 插入间隔
	\begin{subfigure}[b]{0.49\textwidth}\centering
		\includegraphics[width=\textwidth]{./Graphs/example/example.jpeg}
		\caption{sub4}\label{subf::sub4}
	\end{subfigure}
\end{figure}
\begin{figure}[!ht]\ContinuedFloat \centering
	\begin{subfigure}[b]{0.49\textwidth}\centering
		\includegraphics[width=\textwidth]{./Graphs/example/example.jpeg}
		\caption{sub5}\label{subf::sub5}
	\end{subfigure}\hfill % 插入间隔
	\begin{subfigure}[b]{0.49\textwidth}\centering
		\includegraphics[width=\textwidth]{./Graphs/example/example.jpeg}
		\caption{sub6}\label{subf::sub6}
	\end{subfigure}
\end{figure}
\begin{figure}[!ht]\ContinuedFloat \centering
	\begin{subfigure}[b]{0.49\textwidth}\centering
		\includegraphics[width=\textwidth]{./Graphs/example/example.jpeg}
		\caption{sub7}\label{subf::sub7}
	\end{subfigure}\hfill % 插入间隔
	\begin{subfigure}[b]{0.49\textwidth}\centering
		\includegraphics[width=\textwidth]{./Graphs/example/example.jpeg}
		\caption{sub8}\label{subf::sub8}
	\end{subfigure}
\end{figure}
\begin{figure}[!ht]\ContinuedFloat \centering
	\begin{subfigure}[b]{0.49\textwidth}\centering
		\includegraphics[width=\textwidth]{./Graphs/example/example.jpeg}
		\caption{sub9}\label{subf::sub9}
	\end{subfigure}\hfill % 插入间隔
	\begin{subfigure}[b]{0.49\textwidth}\centering
		\includegraphics[width=\textwidth]{./Graphs/example/example.jpeg}
		\caption{sub10}\label{subf::sub10}
	\end{subfigure}
\end{figure}
\begin{figure}[!ht]\ContinuedFloat \centering
	\begin{subfigure}[b]{0.7\textwidth}\centering
		\includegraphics[width=\textwidth]{./Graphs/example/example_2.jpeg}
		\caption{sub11}\label{subf::sub11}
	\end{subfigure}
	\caption{双列跨页子图}
	\label{fig::dual_col_subfigs}
\end{figure}
\FloatBarrier


图~\ref{fig::dual_col_subfigs}\subref{subf::sub1}和图~\ref{fig::dual_col_subfigs}\subref{subf::sub3}分别为XXXXXXXXXXXXXX。

\section{表格}

\subsection{简单表格}

	如表~\ref{tab::table_demo_1}所示。
	\begin{table}[htbp]
		\renewcommand{\arraystretch}{1.5}
		\centering
		\zihao{5}
		\caption{XXXXXXXXXXXXX表}
		\begin{tabular}{c c p{10pt} c c} 
			\toprule % 表头上方线
			表头 & 表头 	&& 表头 & 表头\\ 
			\midrule % 表头与内容间线
			XXX& XXX &&  XXX & XXX\\ 
			XXX& XXX	 && XXX & XXX \\
			XXX& XXX 	 && XXX & XXX \\
			XXX & XXX	 && XXX & XXX \\
			XXX& XXX	 && XXX & XXX \\
			XXX & XXX	 && XXX & XXX \\
			XXX  & 	XXX && XXX & XXX \\
			XXX &  XXX	 && XXX & XXX \\
	 		XXX  & XXX	 && XXX &  XXX\\
	 		XXX  & XXX	 && XXX &  XXX\\
	 		XXX	& XXX	 && XXX &  XXX\\
	 		XXX    & XXX	 && XXX & XXX\\
			\bottomrule % 表底线
		\end{tabular}
		\label{tab::table_demo_1}
	\end{table}
	
	\subsection{跨页表格处理}
	
	见符号表
	
\section{公式}
\subsection{行内公式}

行内$a = b + c$公式


\subsection{单行公式}

如式\ref{eq::example}所示。

\begin{equation}\label{eq::example}
	A=\sqrt{AAAA\left(1-\left(\dfrac{AAAA_{AAAA}}{AAAA}\right)^{\dfrac{A^{\prime}-1}{A^{\prime}}}\right)} 
\end{equation}

\subsection{多行公式}

\begin{equation}
	\left\{\begin{array}{l}
		A = \alpha  \\
		B = \beta \\
		O = \omega
	\end{array}\right.
\end{equation}


\subsection{条件公式}
使用\&符号对齐。
\begin{equation}
	\begin{cases}
		case\quad a & , A>1  
		\\ case\quad b & , B>1 
		\\ case\quad c & , C<1
	\end{cases}
\end{equation}
\subsection{矩阵}

\begin{equation}\label{eq::example_tab}
	A=
	\begin{bmatrix}
		0 & 0 & 2.243 & 0 & -0.2575 & 0\\
		0.0021 & -3.771 & 0 & -0.3055 & 0.0002 & 0.1823\\
		0 & 0 & 0 & 0 & -0.9035 &0 \\
		0 & 0 & 0 & -7.8760 & 0 & 4.6860\\
		0 & 0 & 0 & 0 & 1.99 & 0\\
		-12.09 & 0 & 0 & -0.0519 & 0.0004 &-0.0322
	\end{bmatrix}
\end{equation}


\section{列表项}

\subsection{带编号}
	列表项
	
	\begin{enumerate}[label=(\arabic*),listparindent=2em,left = 2em,topsep = 0pt,itemsep = 0pt,parsep= 0pt,partopsep=0pt]
		\item 列表内容列表内容列表内容
		\item 列表内容列表内容列表内容
		\item 列表内容列表内容列表内容
		\item 列表内容列表内容列表内容
		\item 列表内容列表内容列表内容
		\item 列表内容列表内容列表内容
	\end{enumerate}
	

\section{参考文献引用}

\subsection{角标}

某某\cite{ref_thesis}XXXXXXXXXXXX,某某\cite{ref_book}XXXXXXXXXXXX。

\subsection{强调}

根据文献中\parencite{ref_article}的

















